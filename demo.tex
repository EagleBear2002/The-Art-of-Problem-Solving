%!TEX program=xelatex
\documentclass[10pt]{beamer}
\usepackage[UTF8,noindent]{ctexcap}
\usepackage{listings}

\usetheme{metropolis}
\usepackage{appendixnumberbeamer}

\usepackage{booktabs}
\usepackage[scale=2]{ccicons}

\usepackage{pgfplots}
\usepgfplotslibrary{dateplot}

\usepackage{xspace}
\newcommand{\themename}{\textbf{\textsc{metropolis}}\xspace}

\title{Fast Paxos}
\subtitle{Leaderless-TCS}
\date{\today}
\author{熊丘桓}
\institute{NJU-Software Institute}

\begin{document}

\maketitle

\begin{frame}{Content}
    \setbeamertemplate{section in toc}[sections numbered]
    \tableofcontents[hideallsubsections]
\end{frame}

\section{Classic Paxos}

\begin{frame}[fragile]{Definition}
    \begin{itemize}
        \item Quorum:the Majority is further abstracted, requires an intersection between any two Quorums
        \item Round: each proposal of the proposer is represented by a total ordered number. The execution of this numbered Proposal is called "Round".
        \item i-Quorum: the Quorum needed in Round-i
        \item Classic Round: The Round that performs Classic Paxos
        \item Fast Round: If the leader sends the Any message, the follow-up communication is considered a fast round;
              if the leader does not send the Any message and communicates as before, the follow-up behavior is still Classic Round.
    \end{itemize}
\end{frame}

\begin{frame}[fragile]{Participation Role}
    \begin{itemize}
        \item Client/Proposer/Learner: Responsible for proposals and execute them.
        \item Coordinator: The proposer's coordinator, which can be multiple, and the Client proposes through the Coordinator.
        \item Leader: Elect one of the many Coordinators as the leader.
        \item Acceptor: Responsible for voting on Proposal.
    \end{itemize}
\end{frame}

\begin{frame}[fragile]{Classic Paxos}
    \begin{itemize}
        \item Phase 1a: Leader submits proposal to acceptor
        \item Phase 1b: The acceptor responds with the largest proposer number and the chosen value that has already voted
        \item Phase 2a: The leader collects the return value of the acceptor
        \begin{itemize}
            \item Phase 2a.1: If the acceptor has no return value, then freely decide one
            \item Phase 2a.2: If there is a return value, choose the one with the highest proposal number
        \end{itemize}
        \item Phase 2b: The recipient sends the voting result to the Learner
    \end{itemize}
\end{frame}

\begin{frame}[fragile]{Classic Paxos}
    \includegraphics[width=\textwidth]{pic/Classic Paxos.png}
\end{frame}

\section{Making Paxos Fast}

\begin{frame}[fragile]{Algorithm Process}
    \begin{itemize}
        \item Phase1a: Leader submits proposal to acceptor
        \item Phase1b: The acceptor responds with the largest proposer number and chosen value that has already voted
        \item Phase2a: The leader collects the return value of the acceptor
        \begin{itemize}
            \item Phase2a.1: If the acceptor has no return value, send an Any message to the acceptor, and then the acceptor waits for the proposer to submit the value
            \item Phase2a.2: If there is a return value, choose one according to the rules
        \end{itemize}
        \item Phase2b: The recipient sends the voting result to the learner (including the leader)
    \end{itemize}
\end{frame}

\begin{frame}[fragile]{Algorithm Process}
    \includegraphics[width=\textwidth]{pic/Fast Paxos.png}
\end{frame}

\section{Implementation Considerations}

\begin{frame}[fragile]{Collision}
    \includegraphics[width=\textwidth]{pic/Fast Paxos-2.png}
    MULTIPLE proposers submitting DIFFERENT values to a SINGLE acceptor at the same time is allowed!
\end{frame}

\begin{frame}[fragile]{O4(v) Algorithm}
    If $v$ is chosen in Fast Round $k$, then there exists a k-Quorum $R$ such that for any Acceptor $a \in Q \cap R$, $v$ is voted, where $Q$ is a Quorum of a Classic Round.

    Therefore, if the following two conditions are guaranteed:
    \begin{enumerate}
        \item Each Acceptor votes for only one Value in the same Fast Round
        \item $Q \cap R_v \cap R_w \ne \emptyset$
    \end{enumerate}
    Then $v$ and $w$ cannot be selected at the same time.
\end{frame}

\begin{frame}[fragile]{O4(v) Algorithm}
    According to the above description, in order to prevent a Fast Round from selecting multiple Values, Quorum needs to meet the following two conditions:
    \begin{itemize}
        \item Any two Classic Quorums intersect
        \item Any Classic Quorum intersects with any two Fast Quorums
    \end{itemize}
\end{frame}

\begin{frame}[fragile]{Calculate Quorum}
    Let’s set the total number of Acceptors to be $N$,\\
    the maximum number of failed Acceptors for Classic Round to be $F$,\\
    and the number of allowed failures of Fast Round to be $E$,\\
    that is, N-F constitutes a Quorum of Classic Round, and N-E constitutes a Quorum of Fast Round.\\
    Let $Q_c$ and $Q_f$ be the Quorum sizes of Classic Round and Fast Round, respectively. After sorting, two lower bound results can be obtained:
    \begin{enumerate}
        \item $N > 2F (N \ge 2F + 1)$
        \item $N > 2E + F (2N \ge 2E + F + 2 ?)$ (Still Confused)
        \item $|Q_c| , |Q_f| \ge N - \left \lceil \frac{N}{3} \right \rceil + 1 \ge \left\lfloor \frac{2N}{3} \right \rfloor + 1$
        \item $|Q_c| \ge N - \left\lceil \frac{N}{2} \right\rceil + 1 = \left\lceil \frac{N}{2} \right\rceil+1$
        \item $|Q_f| \ge N - \left\lceil \frac{N}{4} \right\rceil \ge \left\lceil \frac{3N}{4} \right\rceil$
    \end{enumerate}
\end{frame}

\begin{frame}[fragile]{Collision Recovery}
    Optimization: the Acceptor should also send the value to the Leader when voting, so that the Leader can easily find conflicts. If the leader finds a conflict in Round i, it can easily start Round i+1 and re-execute the Classic Paxos process from Phase1a.
    \includegraphics[width=0.7\textwidth]{pic/Fast Paxos-3.png}
\end{frame}

\begin{frame}[fragile]{Collision Recovery}
    Further optimized: if the Leader restarts Round i+1 and receives the Phase1b message sent by i-Quorum Acceptors, the message clarifies two things:
    \begin{enumerate}
        \item Report the maximum Round and corresponding Value that Acceptor a participates in voting
        \item Promise not to vote on Rounds less than i (and less than i+1)
    \end{enumerate}
    If Acceptor a also participates in the voting of Round i, the Phase1b message of a also clarifies the above two things. \\
    ~\\
    Phase2b of Round i has the same meaning as Phase1b of Round i+1 !
\end{frame}

\begin{frame}[fragile]{Collision Recovery}
    \begin{minipage}[t]{0.48\textwidth}
        \centering
        \includegraphics[width=\textwidth]{pic/Fast Paxos-3.png}
        \includegraphics[width=\textwidth]{pic/Fast Paxos-4.png}
        \caption{Successful Submit}
    \end{minipage}
    \begin{minipage}[t]{0.48\textwidth}
        \centering
        \includegraphics[width=\textwidth]{pic/Fast Paxos-5.png}
        \includegraphics[width=\textwidth]{pic/Fast Paxos-3.png}
        \caption{Collision Found}
    \end{minipage}
\end{frame}

\begin{frame}[fragile]{Summary}
    Optimistic locking: find and compensate collisions. The Leader plays a role in initializing the Progress and resolving conflicts. If the Progress has been executed well, the Leader will never participate in the consistency process.\\
    ~\\
    Fast Paxos only needs 2 communication steps in theory, while Classic Paxos needs 3, but Fast Paxos needs at least 1 communication step when resolving conflicts. In high concurrency scenarios, the probability of conflict will be very high, and conflict resolution costs will also be high.\\
    ~\\
    Fast Paxos deeply introduces Client into the algorithm, so that its architecture is far less clear than Classic Paxos, and it is not as easy to extend as Classic Paxos.
\end{frame}

\begin{frame}[standout]
    Thanks for your attention.
\end{frame}

\end{document}


%!TEX program=xelatex
\documentclass[10pt]{beamer}
\usepackage[UTF8,noindent]{ctexcap}
\usepackage{listings}

\usetheme{metropolis}
\usepackage{appendixnumberbeamer}

\usepackage{booktabs}
\usepackage[scale=2]{ccicons}

\usepackage{pgfplots}
\usepgfplotslibrary{dateplot}

\usepackage{xspace}
\newcommand{\themename}{\textbf{\textsc{metropolis}}\xspace}

\title{解决问题的艺术}
\subtitle{The Art of Problem Solving}
\date{\today}
\author{熊丘桓}
\institute{南京大学微软俱乐部}
% \titlegraphic{\hfill\includegraphics[height=1.5cm]{logo.pdf}}

\begin{document}

\maketitle

\begin{frame}{目录}
    \setbeamertemplate{section in toc}[sections numbered]
    \tableofcontents[hideallsubsections]
\end{frame}


\begin{frame}[fragile]{Self Introduction}
    熊丘桓,软件学院2020级本科生。

    现任微软俱乐部技术部副部长。

    任软件学院C程序设计基础助教、开甲书院朋辈导师(以及招生志愿者)。

    深谙“解决问题”艺术。
\end{frame}

\section{提问之前:初探疑云}

\begin{frame}[fragile]{提问之前:初探疑云}
    在您准备要通过电子邮件、新闻群组或者聊天室提出技术问题前,请先做到以下事情:
    \begin{itemize}
        \item 他山之石:尝试在搜索引擎、论坛的旧文章中搜索答案。
        \item 官方预备:尝试阅读手册、常见问题文件(FAQ)以找到答案。
        \item 自力更生:尝试自己检查或试验以找到答案;如果您是程序开发者,请尝试阅读源代码以找到答案。
        \item 向你身边的强者朋友打听以找到答案。
    \end{itemize}

    当你提出问题的时候,请先表明你已经做了上述的努力;这将有助于树立你并不是一个不劳而获且浪费别人的时间的提问者。如果你能一并表达在做了上述努力的过程中所学到的东西会更好,因为我们更乐于回答那些表现出能从答案中学习的人的问题。
\end{frame}

\begin{frame}[fragile]{提问之前:初探疑云}
    尝试上网搜索
    \begin{itemize}
        \item 内事问百度(但很多广告扰乱搜索结果)
        \item 外事、技术问谷歌(重点来了,怎么访问?
        \item 必应介于二者之间
        \item Stack Exchange Community
              \begin{itemize}
                  \item Super User 通用的电脑问题
                  \item Stack Overflow 程序有关问题
                  \item Server Fault 服务器和网管相关问题
              \end{itemize}
        \item 博客:博客园(cnblogs,近期访问困难),CSDN,开源中国
        \item 工具轮子:GITHUB,码云
        \item Linux相关:askubuntu,archwiki
        \item 此外:知乎,思否(SegmentFault),阿里云,腾讯云
    \end{itemize}
\end{frame}

\begin{frame}[fragile]{提问之前:初探疑云}
    尝试自己检查或试验以找到答案
    \begin{itemize}
        \item 查bug原因:
              \begin{itemize}
                  \item 断点调试,看具体变量
                        \begin{itemize}
                            \item Visual Studio
                            \item JetBrains全家桶,有学生免费账户:https://www.jetbrains.com/student/
                        \end{itemize}
                  \item print/alert/assert大法
                  \item Machine is ALWAYS RIGHT!
              \end{itemize}
        \item 确认了bug原因:
              \begin{itemize}
                  \item 复制关键信息,上网搜索。如“ArrayOutOfBoundException”
              \end{itemize}
    \end{itemize}
\end{frame}

\begin{frame}[fragile]{提问之前:初探疑云}
    尝试在你准备提问的论坛的旧文章中搜索答案

    尝试阅读手册(Manual)以找到答案

    尝试阅读FAQ(Frequently Asked Questions)以找到答案

    向你身边的dalao打听以找到答案
\end{frame}


\section{当提问时:切中肯綮}

\begin{frame}{简介}
    拋出的一个技术问题最终是否能得到有用的回答,往往与你提问和追问的方式有较大关联

    黑客们(或大佬们)喜爱有挑战性的问题,或者能激发他们思维的好问题;黑客们有着对那些不愿思考、或者在发问前不做他们该做的事的人的蔑视

    能立刻得到快速并有效答案的最好方法,就是聪明、自信、有解决问题的思路,只是偶尔在特定的问题上需要获得一点帮助
\end{frame}

\begin{frame}{慎选提问的论坛}
    小心选择你要提问的场合,不要:

    \begin{itemize}
        \item 在与主题不合的论坛上贴出你的问题
        \item 在探讨进阶技术问题的论坛张贴非常初级的问题;反之亦然
        \item 在太多的不同群组上重复转贴同样的问题(cross-post)
        \item 向既非熟人也没有义务解决你问题的人发送私人邮件
    \end{itemize}

    用Google找到与你遭遇到困难的软硬件问题最相关的网站。通常那儿都有FAQ、邮件列表及相关说明文件的链接

    提问前在群组或邮件列表的历史记录中搜索与问题相关的关键词

    依照个人经验,大部分的问题都可以通过搜索得到指引或解决
\end{frame}

{
\metroset{titleformat frame=allsmallcaps}
\begin{frame}{All small caps}
    This frame uses the \texttt{allsmallcaps} title format.

    \begin{alertblock}{Potential problems}
        As this title format also uses small caps you face the same problems as with the \texttt{smallcaps} title format. Additionally this format can cause some other problems. Please refer to the documentation if you consider using it.

        As a rule of thumb: just use it for plaintext-only titles.
    \end{alertblock}
\end{frame}
}

{
\metroset{titleformat frame=allcaps}
\begin{frame}{All caps}
    This frame uses the \texttt{allcaps} title format.

    \begin{alertblock}{Potential Problems}
        This title format is not as problematic as the \texttt{allsmallcaps} format, but basically suffers from the same deficiencies. So please have a look at the documentation if you want to use it.
    \end{alertblock}
\end{frame}
}

\section{Elements}

\begin{frame}[fragile]{Typography}
    \begin{verbatim}The theme provides sensible defaults to
\emph{emphasize} text, \alert{accent} parts
or show \textbf{bold} results.\end{verbatim}

    \begin{center}becomes\end{center}

    The theme provides sensible defaults to \emph{emphasize} text,
    \alert{accent} parts or show \textbf{bold} results.
\end{frame}

\begin{frame}{Font feature test}
    \begin{itemize}
        \item Regular
        \item \textit{Italic}
        \item \textsc{Small Caps}
        \item \textbf{Bold}
        \item \textbf{\textit{Bold Italic}}
        \item \textbf{\textsc{Bold Small Caps}}
        \item \texttt{Monospace}
        \item \texttt{\textit{Monospace Italic}}
        \item \texttt{\textbf{Monospace Bold}}
        \item \texttt{\textbf{\textit{Monospace Bold Italic}}}
    \end{itemize}
\end{frame}

\begin{frame}{Lists}
    \begin{columns}[T,onlytextwidth]
        \column{0.33\textwidth}
        Items
        \begin{itemize}
            \item Milk \item Eggs \item Potatoes
        \end{itemize}

        \column{0.33\textwidth}
        Enumerations
        \begin{enumerate}
            \item First, \item Second and \item Last.
        \end{enumerate}

        \column{0.33\textwidth}
        Descriptions
        \begin{description}
            \item[PowerPoint] Meeh. \item[Beamer] Yeeeha.
        \end{description}
    \end{columns}
\end{frame}
\begin{frame}{Animation}
    \begin{itemize}[<+- | alert@+>]
        \item \alert<4>{This is\only<4>{ really} important}
        \item Now this
        \item And now this
    \end{itemize}
\end{frame}
\begin{frame}{Figures}
    \begin{figure}
        \newcounter{density}
        \setcounter{density}{20}
        \begin{tikzpicture}
            \def\couleur{alerted text.fg}
            \path[coordinate] (0,0)  coordinate(A)
            ++( 90:5cm) coordinate(B)
            ++(0:5cm) coordinate(C)
            ++(-90:5cm) coordinate(D);
            \draw[fill=\couleur!\thedensity] (A) -- (B) -- (C) --(D) -- cycle;
            \foreach \x in {1,...,40}{%
                    \pgfmathsetcounter{density}{\thedensity+20}
                    \setcounter{density}{\thedensity}
                    \path[coordinate] coordinate(X) at (A){};
                    \path[coordinate] (A) -- (B) coordinate[pos=.10](A)
                    -- (C) coordinate[pos=.10](B)
                    -- (D) coordinate[pos=.10](C)
                    -- (X) coordinate[pos=.10](D);
                    \draw[fill=\couleur!\thedensity] (A)--(B)--(C)-- (D) -- cycle;
                }
        \end{tikzpicture}
        \caption{Rotated square from
            \href{http://www.texample.net/tikz/examples/rotated-polygons/}{texample.net}.}
    \end{figure}
\end{frame}
\begin{frame}{Tables}
    \begin{table}
        \caption{Largest cities in the world (source: Wikipedia)}
        \begin{tabular}{@{} lr @{}}
            \toprule
            City        & Population \\
            \midrule
            Mexico City & 20,116,842 \\
            Shanghai    & 19,210,000 \\
            Peking      & 15,796,450 \\
            Istanbul    & 14,160,467 \\
            \bottomrule
        \end{tabular}
    \end{table}
\end{frame}
\begin{frame}{Blocks}
    Three different block environments are pre-defined and may be styled with an
    optional background color.

    \begin{columns}[T,onlytextwidth]
        \column{0.5\textwidth}
        \begin{block}{Default}
            Block content.
        \end{block}

        \begin{alertblock}{Alert}
            Block content.
        \end{alertblock}

        \begin{exampleblock}{Example}
            Block content.
        \end{exampleblock}

        \column{0.5\textwidth}

        \metroset{block=fill}

        \begin{block}{Default}
            Block content.
        \end{block}

        \begin{alertblock}{Alert}
            Block content.
        \end{alertblock}

        \begin{exampleblock}{Example}
            Block content.
        \end{exampleblock}

    \end{columns}
\end{frame}
\begin{frame}{Math}
    \begin{equation*}
        e = \lim_{n\to \infty} \left(1 + \frac{1}{n}\right)^n
    \end{equation*}
\end{frame}
\begin{frame}{Line plots}
    \begin{figure}
        \begin{tikzpicture}
            \begin{axis}[
                    mlineplot,
                    width=0.9\textwidth,
                    height=6cm,
                ]

                \addplot {sin(deg(x))};
                \addplot+[samples=100] {sin(deg(2*x))};

            \end{axis}
        \end{tikzpicture}
    \end{figure}
\end{frame}
\begin{frame}{Bar charts}
    \begin{figure}
        \begin{tikzpicture}
            \begin{axis}[
                    mbarplot,
                    xlabel={Foo},
                    ylabel={Bar},
                    width=0.9\textwidth,
                    height=6cm,
                ]

                \addplot plot coordinates {(1, 20) (2, 25) (3, 22.4) (4, 12.4)};
                \addplot plot coordinates {(1, 18) (2, 24) (3, 23.5) (4, 13.2)};
                \addplot plot coordinates {(1, 10) (2, 19) (3, 25) (4, 15.2)};

                \legend{lorem, ipsum, dolor}

            \end{axis}
        \end{tikzpicture}
    \end{figure}
\end{frame}
\begin{frame}{Quotes}
    \begin{quote}
        Veni, Vidi, Vici
    \end{quote}
\end{frame}

{%
\setbeamertemplate{frame footer}{My custom footer}
\begin{frame}[fragile]{Frame footer}
    \themename defines a custom beamer template to add a text to the footer. It can be set via
    \begin{verbatim}\setbeamertemplate{frame footer}{My custom footer}\end{verbatim}
\end{frame}
}

\begin{frame}{References}
    Some references to showcase [allowframebreaks] \cite{knuth92,ConcreteMath,Simpson,Er01,greenwade93}
\end{frame}

\section{Conclusion}

\begin{frame}{Summary}

    Get the source of this theme and the demo presentation from

    \begin{center}\url{github.com/matze/mtheme}\end{center}

    The theme \emph{itself} is licensed under a
    \href{http://creativecommons.org/licenses/by-sa/4.0/}{Creative Commons
        Attribution-ShareAlike 4.0 International License}.

    \begin{center}\ccbysa\end{center}

\end{frame}

\begin{frame}[standout]
    Questions?
\end{frame}

\appendix

\begin{frame}[fragile]{Backup slides}
    Sometimes, it is useful to add slides at the end of your presentation to
    refer to during audience questions.

    The best way to do this is to include the \verb|appendixnumberbeamer|
    package in your preamble and call \verb|\appendix| before your backup slides.

    \themename will automatically turn off slide numbering and progress bars for
    slides in the appendix.
\end{frame}

\begin{frame}[allowframebreaks]{References}

    \bibliography{demo}
    \bibliographystyle{abbrv}

\end{frame}

\end{document}
